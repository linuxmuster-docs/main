\documentclass[ngerman,a4paper]{scrartcl}
\usepackage[T1]{fontenc}
\usepackage[utf8]{inputenc}
\usepackage{booktabs}
\usepackage{tabularx}
\usepackage[scaled=.92]{helvet}
\usepackage{courier}
\renewcommand*{\familydefault}{phv}
\usepackage[left=15mm,top=15mm,bottom=10mm,right=10mm,ignorefoot]{geometry}
\usepackage{graphicx}
\usepackage{color}
\usepackage[
  pdftex,colorlinks=true,
  pdftitle={Checklist Installation linuxmuster.net},pdfsubject={Checklist Installation linuxmuster.net},
  pdfauthor={Sven Röhrauer},
  pdfpagemode=UseNone,pdfstartview=FitH,
  pagebackref,pdfhighlight={/N}
]{hyperref}

\begin{document}
\section*{Installation von linuxmuster.net}
\begin{Form}
\begin{tabular}[t]{@{}llll}
\multicolumn{2}{@{}l}{\textbf{Vorbereitungen}}\\
Datum der Installation &\TextField[%
name=date-of-install,width=20em,bordercolor={0.65 0.79 0.94}]{}\\
Download der Appliances für Firewall/Server/Docker/Opsi &
\CheckBox[name=downloaded-appliances,width=1em,height=0.8em,bordercolor={0.65 0.79 0.94}]{}\\
Netzwerkadressen (wie 10.16.0.0/12 oder 10.0.0.0/16) &
\TextField[name=ipadress-range,width=20em,height=0.8em,bordercolor={0.65 0.79 0.94}]{}\\
\multicolumn{2}{c}{}\\
\multicolumn{2}{@{}l}{\textbf{Installation der Firewall IPFire}}\\
Hostname (Empfehlung: ipfire)&\TextField[%
name=ipfire-local-domain,width=20em,bordercolor={0.65 0.79 0.94}]{}\\
lokale Domäne (Empfehlung: linuxmuster-net.lokal)&\TextField[%
name=name,width=20em,bordercolor={0.65 0.79 0.94}]{}\\
Kennwort des Benutzer „root“&\TextField[%
name=ipfire-password-root,width=20em,bordercolor={0.65 0.79 0.94}]{}\\
Kennwort des Benutzer „admin“&\TextField[%
name=ipfire-password-admin,width=20em,bordercolor={0.65 0.79 0.94}]{}\\
Adresseinstellung IP grün&\ChoiceMenu[print,combo,name=ipfire-adress-green,default=,width=20em,bordercolor={0.65 0.79 0.94}]{}{10.16.1.254, 10.32.1.254, 10.48.1.254, 10.64.1.254, 10.80.1.254, 10.96.1.254, 10.112.1.254, 10.128.1.254, 10.144.1.254, 10.160.1.254, 10.176.1.254, 10.192.1.254, 10.208.1.254, 10.224.1.254}\\
Adresseinstellung IP rot&\ChoiceMenu[print,name=adress-red,combo,default= ,width=20em,bordercolor={0.65 0.79 0.94}]{}{DHCP, statische IP}\\
\hspace*{1cm} falls statische IP: DNS1&\TextField[%
name=ipfire-dns1,width=20em,bordercolor={0.65 0.79 0.94}]{}\\
\hspace*{1cm} falls statische IP: DNS2&\TextField[%
name=ipfire-dns2,width=20em,bordercolor={0.65 0.79 0.94}]{}\\
\hspace*{1cm} falls statische IP: gateway&\TextField[%
name=ipfire-gateway,width=20em,bordercolor={0.65 0.79 0.94}]{}\\
Adresseinstellung IP blau&\ChoiceMenu[name=adress-blue,print,combo,default=,width=20em,bordercolor={0.65 0.79 0.94}]{}{172.16.16.254, 172.16.31.254, 172.16.48.254, 172.16.64.254, 172.16.80.254, 172.16.96.254, 172.16.112.254, 172.16.128.254, 172.16.144.254, 172.16.160.254, 172.16.176.254, 172.16.192.254, 172.16.208.254, 172.16.224.254}\\
\multicolumn{2}{c}{}\\
\multicolumn{2}{@{}l}{\textbf{Konfiguration der Firewall IPFire}}\\
ssh-Zugriff aktiviert & \CheckBox[name=ipfire-ssh-enabled,width=1em,height=0.8em,bordercolor={0.65 0.79 0.94}]{}\\
Uneingeschränkten Proxyzugriff für den Server eingestellt & \CheckBox[name=ipfire-proxy-server,width=1em,height=0.8em,bordercolor={0.65 0.79 0.94}]{}\\
\multicolumn{2}{c}{}\\
\multicolumn{2}{@{}l}{\textbf{Installation des linuxmuster.net-Servers}}\\
IP-Adresse des Servers&\ChoiceMenu[print,combo,name=server-ip,default=,width=20em,bordercolor={0.65 0.79 0.94}]{}{10.16.1.1, 10.32.1.1, 10.48.1.1, 10.64.1.1, 10.80.1.1, 10.96.1.1, 10.112.1.1, 10.128.1.1, 10.144.1.1, 10.160.1.1, 10.176.1.1, 10.192.1.1, 10.208.1.1, 10.224.1.1}\\
DNS-Server&\ChoiceMenu[print,combo,name=ipfire-adress-green,default=,width=20em,bordercolor={0.65 0.79 0.94}]{}{10.16.1.254, 10.32.1.254, 10.48.1.254, 10.64.1.254, 10.80.1.254, 10.96.1.254, 10.112.1.254, 10.128.1.254, 10.144.1.254, 10.160.1.254, 10.176.1.254, 10.192.1.254, 10.208.1.254, 10.224.1.254}\\
Gateway&\ChoiceMenu[print,combo,name=ipfire-adress-green,default=,width=20em,bordercolor={0.65 0.79 0.94}]{}{10.16.1.254, 10.32.1.254, 10.48.1.254, 10.64.1.254, 10.80.1.254, 10.96.1.254, 10.112.1.254, 10.128.1.254, 10.144.1.254, 10.160.1.254, 10.176.1.254, 10.192.1.254, 10.208.1.254, 10.224.1.254}\\
Hostname&\TextField[%
name=server-hostname,width=20em,bordercolor={0.65 0.79 0.94}]{}\\
Hostname&\TextField[%
name=server-external-name,width=20em,bordercolor={0.65 0.79 0.94}]{}\\
Name des administrativen Nutzers&\TextField[%
name=server-name-admin,width=20em,bordercolor={0.65 0.79 0.94}]{}\\
Passwort des administrativen Nutzers&\TextField[%
name=server-password-admin,width=20em,bordercolor={0.65 0.79 0.94}]{}\\
\multicolumn{2}{c}{}\\
\multicolumn{2}{@{}l}{\textbf{linuxmuster.net Pakete installieren}}\\
Paketquellen eintragen &
\CheckBox[name=lmn-package,width=1em,height=0.8em,bordercolor={0.65 0.79 0.94}]{}\\
Installieren mit: \\
\texttt{sudo apt-get install linuxmuster-base}  &
\CheckBox[name=lmn-install,width=1em,height=0.8em,bordercolor={0.65 0.79 0.94}]{}\\
\multicolumn{2}{c}{}\\
\multicolumn{2}{@{}l}{\textbf{linuxmuster.net konfigurieren}}\\
Konfiguration starten mit:\\
\texttt{sudo linuxmuster-setup -first}  & \CheckBox[name=lmn-config,width=1em,height=0.8em,bordercolor={0.65 0.79 0.94}]{}\\
Länderkürzel&\ChoiceMenu[print,combo,default=,width=20em,bordercolor={0.65 0.79 0.94}]{}{DE, AT, CH, --}\\
Bundesland&\ChoiceMenu[print,combo,default=,width=20em,bordercolor={0.65 0.79 0.94}]{}{BW, BY, BE, BB, HB, HH, HE, MV, NI, NW, RP, SL, SN, ST, SH, TH, --}\\
Schulort&\TextField[%
name=lmn-school-place,width=20em,bordercolor={0.65 0.79 0.94}]{}\\
Schulname&\TextField[%
name=lmn-school-name,width=20em,bordercolor={0.65 0.79 0.94}]{}\\
Schulort&\TextField[%
name=lmn-school-place,width=20em,bordercolor={0.65 0.79 0.94}]{}\\
Samba-Domäne&\TextField[%
name=lmn-school-place,width=20em,bordercolor={0.65 0.79 0.94}]{}\\
Hostname&\TextField[%
name=lmn-hostname,width=20em,bordercolor={0.65 0.79 0.94}]{}\\
lokale Domäne&\TextField[%
name=lmn-domain,width=20em,bordercolor={0.65 0.79 0.94}]{}\\
IP-Bereich&\ChoiceMenu[print,combo,default= ,width=20em,bordercolor={0.65 0.79 0.94}]{}{16-31, 32-47, 48-63, 64-79, 80-95, 96-111, 112-127, 128-143, 144-159, 160-175, 176-191, 192-207,208-223,224-239}\\
externer Servername&\TextField[%
name=lmn-external-name,width=20em,bordercolor={0.65 0.79 0.94}]{}\\
SMTP-Relay-Host&\TextField[%
name=lmn-smtp,width=20em,bordercolor={0.65 0.79 0.94}]{}\\
subnetting&\TextField[%
name=lmn-subnetting,width=20em,bordercolor={0.65 0.79 0.94}]{}\\
Passwort des Benutzers administrator&\TextField[%
name=lmn-admin-password,width=20em,bordercolor={0.65 0.79 0.94}]{}\\
Neustart mit \texttt{sudo reboot}  &\CheckBox[name=lmn-reboot,width=1em,height=0.8em,bordercolor={0.65 0.79 0.94}]{}\\
\end{tabular}
\end{Form}
\end{document}

 
